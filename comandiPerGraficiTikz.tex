\usepackage{graphicx}
\usepackage{float}
\usepackage{sidecap}
\usepackage{amsmath}
\usepackage{pgfplots}   % per i grafici
\pgfplotsset{width=7cm,compat=1.9}
\usepackage{booktabs}
\usepackage{amssymb}





%disegni
\usepackage{tikz}
\usetikzlibrary{patterns, shapes.geometric}
\usepackage{physics}
\usepackage{ifthen}
\usepackage{tikz}
\usepackage[outline]{contour} % glow around text
\usetikzlibrary{calc} % for pic
\usetikzlibrary{angles,quotes} % for pic
\usetikzlibrary{patterns}
\tikzset{>=latex} % for LaTeX arrow head
\contourlength{1.2pt}

\colorlet{myred}{red!65!black}
\tikzstyle{ground}=[preaction={fill,top color=black!10,bottom color=black!5,shading angle=20},
                    fill,pattern=north east lines,draw=none,minimum width=0.3,minimum height=0.6]
\tikzstyle{mass}=[line width=0.6,red!30!black,fill=red!40!black!10,rounded corners=1,
                  top color=red!40!black!20,bottom color=red!40!black!10,shading angle=20]
\tikzstyle{rope}=[brown!70!black,line width=1.2,line cap=round] %very thick

% FORCES SWITCH
\tikzstyle{force}=[->,myred,thick,line cap=round]
\tikzstyle{Fproj}=[force,myred!40]
\newcommand{\vbF}{\vb{F}}
\newboolean{showforces}
\setboolean{showforces}{true}




%prodotto scalare 
\usepackage{amsmath}
\usepackage{tikz}
\usepackage{physics}
\usepackage[outline]{contour} % glow around text
\usetikzlibrary{angles,quotes} % for pic
\contourlength{1.2pt}

\tikzset{>=latex} % for LaTeX arrow head
\usepackage{xcolor}
\colorlet{veccol}{green!70!black}
\colorlet{vcol}{green!70!black}
\colorlet{xcol}{blue!85!black}
\colorlet{projcol}{xcol!60}
\colorlet{unitcol}{xcol!60!black!85}
\colorlet{myblue}{blue!70!black}
\colorlet{myred}{red!90!black}
\colorlet{mypurple}{blue!50!red!80!black!80}
\tikzstyle{vector}=[->,very thick,xcol]


%integrale lavoro
\usetikzlibrary{calc} % for pic
\usetikzlibrary{arrows.meta}
\usetikzlibrary{patterns}
\usetikzlibrary{angles,quotes} % for pic
\tikzset{>=latex} % for LaTeX arrow head

\colorlet{myred}{red!65!black}
%\colorlet{mylightblue}{blue!20}
\colorlet{mydarkblue}{blue!30!black}
\colorlet{xcol}{blue!70!black}
\colorlet{vcol}{green!70!black}
\colorlet{acol}{red!50!blue!80!black!80}
\tikzstyle{ground}=[preaction={fill,top color=black!10,bottom color=black!5,shading angle=20},
                    fill,pattern=north east lines,draw=none,minimum width=0.3,minimum height=0.6]
\tikzstyle{mass}=[line width=0.6,red!30!black,fill=red!40!black!10,rounded corners=1,
                  top color=red!40!black!20,bottom color=red!40!black!10,shading angle=20]
\tikzstyle{vector}=[->,very thick,xcol,line cap=round]
\tikzstyle{force}=[->,myred,thick,line cap=round]
\tikzstyle{Fproj}=[force,myred!40]
\tikzstyle{mydashed}=[dash pattern=on 2pt off 2pt]
\tikzstyle{smallarrow}=[{Latex[length=2,width=2]}-{Latex[length=2,width=2]}]
\def\tick#1#2{\draw[thick] (#1) ++ (#2:0.1) --++ (#2-180:0.2)} %0.03*\xmax


%molle
\usepackage[outline]{contour} % glow around text
\usetikzlibrary{patterns,snakes}
\usetikzlibrary{arrows.meta} % for arrow size
\contourlength{0.4pt}

\colorlet{xcol}{blue!70!black}
\colorlet{darkblue}{blue!40!black}
\colorlet{myred}{red!65!black}
\tikzstyle{mydashed}=[xcol,dashed,line width=0.25,dash pattern=on 2.2pt off 2.2pt]
\tikzstyle{axis}=[->,thick] %line width=0.6
\tikzstyle{ell}=[{Latex[length=3.3,width=2.2]}-{Latex[length=3.3,width=2.2]},line width=0.3]
\tikzstyle{dx}=[-{Latex[length=3.3,width=2.2]},darkblue,line width=0.3]
\tikzstyle{ground}=[preaction={fill,top color=black!10,bottom color=black!5,shading angle=20},
                    fill,pattern=north east lines,draw=none,minimum width=0.3,minimum height=0.6]
\tikzstyle{mass}=[line width=0.6,red!30!black,fill=red!40!black!10,rounded corners=1,
                  top color=red!40!black!20,bottom color=red!40!black!10,shading angle=20]
\tikzstyle{spring}=[line width=0.8,blue!7!black!80,snake=coil,segment amplitude=5,segment length=5,line cap=round]
\tikzset{>=latex} % for LaTeX arrow head
\tikzstyle{force}=[->,myred,very thick,line cap=round]
\def\tick#1#2{\draw[thick] (#1)++(#2:0.1) --++ (#2-180:0.2)}

%pensolo
\usepackage{siunitx}
\usepackage{tikz,pgfplots}
\usepackage[outline]{contour} % glow around text
\usetikzlibrary{calc}
\usetikzlibrary{angles,quotes} % for pic
\usetikzlibrary{arrows.meta}
\tikzset{>=latex} % for LaTeX arrow head
\contourlength{1.2pt}

\colorlet{xcol}{blue!70!black}
\colorlet{vcol}{green!60!black}
\colorlet{myred}{red!70!black}
\colorlet{myblue}{blue!70!black}
\colorlet{mygreen}{green!70!black}
\colorlet{mydarkred}{myred!70!black}
\colorlet{mydarkblue}{myblue!60!black}
\colorlet{mydarkgreen}{mygreen!60!black}
\colorlet{acol}{red!50!blue!80!black!80}
\tikzstyle{CM}=[red!40!black,fill=red!80!black!80]
\tikzstyle{xline}=[xcol,thick,smooth]
\tikzstyle{mass}=[line width=0.6,red!30!black,fill=red!40!black!10,rounded corners=1,
                  top color=red!40!black!20,bottom color=red!40!black!10,shading angle=20]
\tikzstyle{faded mass}=[dashed,line width=0.1,red!30!black!40,fill=red!40!black!10,rounded corners=1,
                        top color=red!40!black!10,bottom color=red!40!black!10,shading angle=20]
\tikzstyle{rope}=[brown!70!black,very thick,line cap=round]
\def\rope#1{ \draw[black,line width=1.4] #1; \draw[rope,line width=1.1] #1; }
\tikzstyle{force}=[->,myred,very thick,line cap=round]
\tikzstyle{velocity}=[->,vcol,very thick,line cap=round]
\tikzstyle{Fproj}=[force,myred!40]
\tikzstyle{myarr}=[-{Latex[length=3,width=2]},thin]
\def\tick#1#2{\draw[thick] (#1)++(#2:0.12) --++ (#2-180:0.24)}
\DeclareMathOperator{\sn}{sn}
\DeclareMathOperator{\cn}{cn}
\DeclareMathOperator{\dn}{dn}
\def\N{80} % number of samples in plots


%urti
\usepackage{siunitx}
\usepackage{tikz}
\usetikzlibrary{calc}
\usetikzlibrary{angles,quotes} % for pic
\usetikzlibrary{patterns}
\tikzset{>=latex} % for LaTeX arrow head

\colorlet{xcol}{blue!70!black}
\colorlet{vcol}{green!60!black}
\colorlet{myred}{red!65!black}
\colorlet{acol}{red!50!blue!80!black!80}
\tikzstyle{mass}=[line width=0.6,red!30!black,fill=red!40!black!10,rounded corners=1,
                  top color=red!40!black!20,bottom color=red!40!black!10,shading angle=20]
\tikzstyle{ground}=[preaction={fill,top color=blue!50!black!10,bottom color=blue!50!black!5,shading angle=20},
                    fill,pattern color=blue!20!black,pattern=north east lines,draw=none,minimum width=0.3,minimum height=0.6]
\tikzstyle{velocity}=[->,vcol,very thick,line cap=round]

\tikzset{
  pics/collision/.style={
    code={
      \draw[line width=0.5*#1,orange,fill=yellow]
        (0:0.20*#1) -- (30:0.06*#1) -- (50:0.25*#1) -- (80:0.10*#1) -- (105:0.32*#1) --
        (140:0.08*#1) -- (170:0.25*#1) -- (190:0.08*#1) -- (220:0.25*#1) --
        (250:0.08*#1) -- (270:0.24*#1) -- (300:0.08*#1) -- (320:0.25*#1) -- (340:0.09*#1) -- cycle;
  }},
  pics/collision/.default=1,
}


%ruota bici
\usepackage{tikz}
\usepackage[outline]{contour} % glow around text
\usetikzlibrary{calc}
\usetikzlibrary{angles,quotes} % for pic
\tikzset{>=latex} % for LaTeX arrow head
\contourlength{1.1pt}

\colorlet{xcol}{blue!98!black}
\colorlet{xcoldark}{blue!50!black}
\colorlet{vcol}{green!70!black}
\colorlet{myred}{red!80!black}
\colorlet{mypurple}{blue!60!red!80}
\colorlet{acol}{red!50!blue!80!black!80}
\tikzstyle{rvec}=[->,xcol,very thick,line cap=round]
\tikzstyle{force}=[->,myred,very thick,line cap=round]
\tikzstyle{mass}=[line width=0.6,red!30!black,fill=red!40!black!10,rounded corners=1,
                  top color=red!40!black!20,bottom color=red!40!black!10,shading angle=20]

\tikzset{
  pics/Tin/.style={
    code={
      \def\R{0.12}
      \draw[pic actions,line width=0.6,#1,fill=white] % ,thick
        (0,0) circle (\R) (-135:.75*\R) -- (45:.75*\R) (-45:.75*\R) -- (135:.75*\R);
  }},
  pics/Tout/.style={
    code={
      \def\R{0.12}
      \draw[pic actions,line width=0.6,#1,fill=white] (0,0) circle (\R);
      \fill[pic actions,#1] (0,0) circle (0.3*\R);
  }},
  pics/Tin/.default=mypurple,
  pics/Tout/.default=mypurple,
}

\newcommand\rightAngle[4]{
  \pgfmathanglebetweenpoints{\pgfpointanchor{#2}{center}}{\pgfpointanchor{#3}{center}}
  \coordinate (tmpRA) at ($(#2)+(\pgfmathresult+45:#4)$);
  \draw[white,line width=0.7] ($(#2)!(tmpRA)!(#1)$) -- (tmpRA) -- ($(#2)!(tmpRA)!(#3)$);
  \draw[xcoldark] ($(#2)!(tmpRA)!(#1)$) -- (tmpRA) -- ($(#2)!(tmpRA)!(#3)$);
}

\def\r{0.16}  % axis radius
\def\Ri{1.18} % wheel rims inside
\def\Rr{1.30} % wheel rims outside
\def\Rt{1.45} % wheel tyre
\def\wheel{
  \def\Ns{11}   % number of spokes
  \def\Nn{34}   % number of tyre stud
  \coordinate (O) at (0,0);
  \foreach \i [evaluate={\ang=\i*360/(\Ns+1);}] in {0,...,\Ns}{
    \draw[line width=0.5,black!70,line cap=round] (\ang-90:\r) --++ (\ang:\Rr);
  }
  \draw[very thin,fill=black!30] (O) circle (1.2*\r);
  \draw[very thin,black!70] (O) circle (0.6*\r) circle (0.3*\r);
  \foreach \i [evaluate={\ang=\i*360/(\Ns+1);}] in {0,...,\Ns}{
    \fill (\ang+90:\r) circle (0.015);
    \draw[line width=0.5,black!70,line cap=round] (\ang+90:\r) --++ (\ang:\Rr);
  }
  \draw[fill=black!30,even odd rule] (O) circle (\Ri) circle (\Rr);
  
  \draw[fill=black!84,even odd rule] (O) circle (\Rr) (\Rt,0)
    \foreach \i [evaluate={\anga=\i*360/(\Nn+1);\angb=(\i+0.5)*360/(\Nn+1);
                           \angc=(\i+1)*360/(\Nn+1);}] in {0,...,\Nn}{
      -- (\anga:1.008*\Rt) arc(\anga:\angb:1.008*\Rt)
      -- (\angb:1.000*\Rt) arc(\angb:\angc:1.000*\Rt)
    } -- cycle;
  \draw[very thin] (O) circle (0.97*\Rt) circle (0.945*\Rt);
}


%regola mano destra

\usetikzlibrary{angles,quotes} % for pic (angle labels)
\usepackage{xcolor}
\colorlet{pinkskin}{pink!25}
\colorlet{brownskin}{pink!5!brown!45}
\colorlet{myred}{red!90!black}
\colorlet{myblue}{blue!90!black}
\colorlet{mypurple}{blue!50!red!80!black!80}
\colorlet{Bcol}{violet!90}
\colorlet{BFcol}{red!60!black}
\colorlet{veccol}{green!45!black}
\colorlet{Icol}{blue!70!black}
\colorlet{mucol}{red!90!black}
\tikzstyle{BField}=[->,line width=2,Bcol]
\tikzstyle{current}=[->,Icol] %thick,
\tikzstyle{force}=[->,line width=2,BFcol]
\tikzstyle{vector}=[->,line width=2,veccol]
\tikzstyle{thick vector}=[->,line width=2,veccol]
\tikzstyle{mu vector}=[->,line width=2,mucol]
\tikzstyle{velocity}=[->,line width=2,veccol]
\tikzstyle{charge+}=[very thin,draw=black,top color=red!50,bottom color=red!90!black,shading angle=20,circle,inner sep=0.5]


%momento angolare
\usetikzlibrary{bending} % for arrow head angle
\colorlet{xcol}{blue!70!black}
\colorlet{vcol}{green!60!black}
\colorlet{pcol}{red!60!black}
\colorlet{Lcol}{green!50!black}
\colorlet{myred}{red!65!black}
\colorlet{mypurple}{blue!60!red!80}
\colorlet{acol}{red!50!blue!80!black!80}
\tikzstyle{rvec}=[->,xcol,very thick,line cap=round]
\tikzstyle{vvec}=[->,vcol,very thick,line cap=round]
\tikzstyle{pvec}=[->,pcol,very thick,line cap=round]
\tikzstyle{Lvec}=[->,Lcol,very thick,line cap=round]
\tikzstyle{mass}=[line width=0.6,red!30!black,draw=red!30!black, %rounded corners=1,
                  top color=red!40!black!30,bottom color=red!40!black!10,shading angle=30]


%momenti di inerzia
\usepackage{physics}
\usepackage{tikz}
\usepackage[outline]{contour} % glow around text
\usetikzlibrary{calc}
\usetikzlibrary{angles,quotes} % for pic
\usetikzlibrary{arrows.meta}
\usetikzlibrary{patterns}
\usetikzlibrary{bending} % for arrow head angle
\tikzset{>=latex} % for LaTeX arrow head
\contourlength{0.8pt}

\colorlet{xcol}{blue!70!black}
\colorlet{myred}{red!65!black}
\tikzstyle{rvec}=[->,xcol,very thick,line cap=round]
\tikzstyle{mass line}=[line width=0.5,draw=red!30!black]
\tikzstyle{myarr}=[-{Latex[length=3,width=2]},blue!40!black]
\tikzstyle{myarr2}=[{Latex[length=3,width=2]}-{Latex[length=3,width=2]},blue!40!black]
\tikzstyle{mass}=[mass line, %rounded corners=1,
                  top color=red!40!black!30,bottom color=red!40!black!10,shading angle=30]
\tikzstyle{middle mass}=[mass line,top color=red!40!black!50,bottom color=red!40!black!50,
                         middle color=red!40!black!10,shading angle=30]

\def\r{0.05} % pulley small radius
\tikzset{
  pics/rotarr/.style={
    code={
      \draw[white,line width=0.8] ({#1*cos(210)},0) arc(-210:35:{#1} and {0.35*#1});
      \draw[-{>[flex'=1]}] ({#1*cos(210)},0) coordinate (W1) arc(-210:35:{#1} and {0.35*#1})
        node[midway] (W2) {} --++ (150:0.1) coordinate (W3);
  }},
  pics/rotarr/.default=0.3,
}


%coriolis e sistema inerziale

\usepackage[outline]{contour} % glow around text
\usetikzlibrary{calc}
\usetikzlibrary{decorations.markings}
\usetikzlibrary{angles,quotes} % for pic
\usetikzlibrary{arrows.meta} % for arrow size
\usetikzlibrary{bending} % for arrow head angle
\usetikzlibrary{decorations.pathmorphing} % for decorate random steps
\tikzset{>=latex} % for LaTeX arrow head
\usepackage{xcolor}
\contourlength{1.3pt}

\colorlet{xcol}{blue!70!black}
\colorlet{xcol'}{xcol!50!red}
\colorlet{vcol}{green!45!black}
\colorlet{acol}{red!50!blue!80!black!80}
\tikzstyle{rvec}=[->,very thick,xcol,line cap=round]
\tikzstyle{vvec}=[->,very thick,vcol,line cap=round]
\tikzstyle{avec}=[->,very thick,acol,line cap=round]
\colorlet{myred}{red!65!black}
\tikzstyle{force}=[->,myred,very thick,line cap=round]
\tikzstyle{mass}=[line width=0.5,draw=red!50!black,fill=red!50!black!10,rounded corners=1,
                  top color=red!50!black!30,bottom color=red!50!black!10,shading angle=20]
\tikzstyle{disk}=[line width=0.5,orange!30!black,fill=orange!40!black!10,
                  top color=orange!40!black!20,bottom color=orange!40!black!10,shading angle=20]
\tikzstyle{pole}=[line width=0.5,blue!20!black,fill=orange!20!black!10,
                  top color=blue!20!black!20,bottom color=blue!20!black!10,shading angle=20]
\tikzstyle{rope}=[brown!70!black,line width=1,line cap=round] %very thick
\tikzstyle{myarr}=[-{Latex[length=3,width=2,flex'=1]},thin]
\tikzstyle{mydashed}=[dash pattern=on 2pt off 1pt]
\def\rope#1{ \draw[rope,black,line width=1.4] #1; \draw[rope,line width=1.1] #1; }
\def\tick#1#2{\draw[thick] (#1) ++ (#2:0.1) --++ (#2-180:0.2)}
\newcommand\rightAnglee[4]{
  \pgfmathanglebetweenpoints{\pgfpointanchor{#2}{center}}{\pgfpointanchor{#3}{center}}
  \coordinate (tmpRA) at ($(#2)+(\pgfmathresult+45:#4)$);
  %\draw[white,line width=0.6] ($(#2)!(tmpRA)!(#1)$) -- (tmpRA) -- ($(#2)!(tmpRA)!(#3)$);
  \draw[black] ($(#2)!(tmpRA)!(#1)$) -- (tmpRA) -- ($(#2)!(tmpRA)!(#3)$);
}


%culomb forza
\tikzstyle{charge}=[thin,top color=red!50,bottom color=red!70,shading angle=20]
\tikzstyle{charge+}=[thin,top color=red!50,bottom color=red!90!black,shading angle=20]
\tikzstyle{charge-}=[thin,top color=blue!50,bottom color=blue!80,shading angle=20]
\tikzstyle{force}=[->,very thick,orange!80!black]
\tikzstyle{vector}=[->,very thick,green!45!black]

%campo elettrico
\usepackage{bm}
\usepackage{tikz,pgfplots}
\tikzset{>=latex} % for LaTeX arrow head
\pgfplotsset{compat=1.13}
\usetikzlibrary{decorations.markings,intersections,calc}
\usepackage[outline]{contour} % glow around text
\usetikzlibrary{angles,quotes} % for pic (angle labels)
\usepackage{xcolor}
\colorlet{Ecol}{orange!90!black}
\colorlet{EcolFL}{orange!80!black}
\colorlet{Bcol}{blue!90!black}
\tikzstyle{EcolEP}=[blue!80!white]
\colorlet{veccol}{green!45!black}
\tikzstyle{charge+}=[very thin,top color=red!50,bottom color=red!90!black,shading angle=20]
\tikzstyle{charge-}=[very thin,top color=blue!50,bottom color=blue!80,shading angle=20]
\tikzstyle{vector}=[->,thick,veccol]
\tikzset{EFieldLineArrow/.style={EcolFL,decoration={markings,mark=at position #1 with {\arrow{latex}}},
                                 postaction={decorate}},
         EFieldLineArrow/.default=0.5}


%campo elettrico interazioni
\pgfplotsset{compat=1.13}
\usetikzlibrary{decorations.markings,intersections,calc}
\usepackage{ifthen}
%\usepackage{etoolbox}
\usepackage{xcolor}
\colorlet{Ecol}{orange!90!black}
\colorlet{EcolFL}{orange!80!black}
\colorlet{Bcol}{blue!90!black}
\tikzstyle{EcolEP}=[blue!80!white]
\tikzstyle{charge+}=[very thin,top color=red!50,bottom color=red!90!black,shading angle=20]
\tikzstyle{charge-}=[very thin,top color=blue!50,bottom color=blue!80,shading angle=20]
%\tikzstyle{EFielLineArrow2}=[
%    EcolFL,decoration={markings,
%          mark=at position 0.5 with {\arrow[rotate=\angle]{latex}}},
%          postaction={decorate}]
\tikzset{
   EFielLineArrow/.style args = {#1}{EcolFL,decoration={markings,
          mark=at position 0.5 with {\arrow[rotate=#1]{latex}}},
          postaction={decorate}}
}

\makeatletter
  \newcommand{\xy}[3]{% % FIND X, Y
    \tikz@scan@one@point\pgfutil@firstofone#1\relax
    \edef#2{\the\pgf@x}%
    \edef#3{\the\pgf@y}%
  }
\makeatother
\newcommand{\EFielLineArrow}[2]{ % ELECTRIC FIELD LINE ARROW
  \pgfkeys{/pgf/fpu,/pgf/fpu/output format=fixed} % for calculation between -1*10^324 and +1*10^324
  \pgfmathsetmacro{\x}{#1/28.45pt}
  \pgfmathsetmacro{\y}{#2/28.45pt}
  \pgfmathsetmacro{\U}{\Q*((\x+\a)^2+(\y)^2)^(3/2)}
  \pgfmathsetmacro{\V}{\q*((\x-\a)^2+(\y)^2)^(3/2)}
  \pgfkeys{/pgf/fpu=false}
  \pgfmathparse{
    atan2(((\y)*\V + (\y)*\U),((\x+\a)*\V + (\x-\a)*\U))
  }
  \edef\angle{\pgfmathresult}
  \pgfmathsetmacro{\D}{int(1000*\q*(\x+\a)/sqrt((\x+\a)^2+\y*\y) + 1000*\Q*(\x-\a)/sqrt((\x-\a)^2+\y*\y))/1000}
  \draw[EFielLineArrow={\angle}] (\xpt,\ypt);
}
\newcommand{\EVector}[2]{ % ELECTRIC FIELD VECTOR
  \pgfmathsetmacro{\x}{#1/28.45pt}
  \pgfmathsetmacro{\y}{#2/28.45pt}
  \pgfmathsetmacro{\U}{((\x+\a)^2+(\y)^2)^(3/2)}
  \pgfmathsetmacro{\V}{((\x-\a)^2+(\y)^2)^(3/2)}
  \pgfmathsetmacro{\Ex}{\q*3*(\x+\a)/\U + \Q*3*(\x-\a)/\V}
  \pgfmathsetmacro{\Ey}{\q*3*(\y)/\U + \Q*3*(\y)/\V}
  \fill[Ecol] (\x,\y) circle (0.8pt);
  \draw[->,Ecol,thick] (\x,\y) --++ (\Ex,\Ey)
    node[Ecol,right,scale=0.8] {$\bm{E}$};
  \pgfmathsetmacro{\D}{int(1000*\q*(\x+\a)/sqrt((\x+\a)^2+\y*\y) + 1000*\Q*(\x-\a)/sqrt((\x-\a)^2+\y*\y))/1000}
}
\newcommand{\EFieldLines}{ % ELECTRIC FIELD LINES
  \message{^^JField lines (\q,\Q) with contours range ^^J\range^^J}
  
  % PATHS for intersections
  \path[name path=ellipse1] (-\a,0) ellipse ({0.90*\R} and {1.5*\R});
  \path[name path=ellipse2] (+\a,0) ellipse ({0.75*\R} and {1.5*\R});
  \path[name path=ellipse3] (  0,0) ellipse ({\a+\R} and 1.5*\R);
  
  % FIELD LINES
  \draw[EcolFL,name path=Elines] plot[id=plot, raw gnuplot, smooth]
    function{
       f(x,y) = \q*(x+\a)/sqrt((x+\a)**2+y**2) + \Q*(x-\a)/sqrt((x-\a)**2+y**2);
       set xrange [\xmin:\xmax];
       set yrange [-\ymax:\ymax];
       set view 0,0;
       set isosample 400,400;
       set cont base;
       set cntrparam levels discrete \range;
       unset surface;
       splot f(x,y)
    };
  
  % ELLIPSE INTERSECTIONS
  \pgfmathsetmacro{\oppositesign}{\q*\Q<0 ? 1 : 0}
  \ifthenelse{\oppositesign>0}{
    % OPPOSITE SIGN
    \foreach \c in {1,2}{
      \message{Intersections \c...}
      \path[name intersections={of=Elines and ellipse\c,total=\t}]
        \pgfextra{\xdef\Nb{\t}};
      \message{ found \Nb ^^J}
      \foreach \i in {1,...,\Nb}{
        \message{  \i}
        \xy{(intersection-\i)}{\xpt}{\ypt}
        \EFielLineArrow{\xpt}{\ypt}
        \message{ (\D,\x,\y)^^J}
      }
    }
  }{
    % SAME SIGN
    \message{Intersections...}
    \path[name intersections={of=Elines and ellipse3,total=\t}]
        \pgfextra{\xdef\Nb{\t}}; 
    \message{ found \Nb ^^J}
    \foreach \i in {1,...,\Nb}{
      \message{  \i}
      \xy{(intersection-\i)}{\xpt}{\ypt}
      \EFielLineArrow{\xpt}{\ypt}
      \message{ (\D,\x,\y)^^J}
    }
  }
}
\newcommand{\EEquipot}{ % EQUIPOTENTIAL SURFACE
  \message{^^JEquipotential surface (\q,\Q) with contours range ^^J\rangeEP}
  
  % FIELD LINES
  \draw[EcolEP] plot[id=plot, raw gnuplot, smooth] %,dashed
    function{
       f(x,y) = \q/sqrt((x+\a)**2+y**2) + \Q/sqrt((x-\a)**2+y**2);
       set xrange [\xmin:\xmax];
       set yrange [-\ymax:\ymax];
       set view 0,0;
       set isosample 400,400;
       set cont base;
       set cntrparam levels discrete \rangeEP;
       unset surface;
       splot f(x,y)
    };
}
\newcommand{\EVectorOnFieldLines}{ % ELECTRIC FIELD VECTOR
  \message{^^JVector on field lines (\q,\Q) with contours range ^^J\C^^J}
  
  % FIELD LINES
  \path[name path=xline] (\Cx,-\ymax) -- (\Cx,-\a) (\Cx,\a) -- (\Cx,\ymax);
  \path[name path=Elines] plot[id=plot, raw gnuplot, smooth]
    function{
       f(x,y) = \q*(x+\a)/sqrt((x+\a)**2+y**2) + \Q*(x-\a)/sqrt((x-\a)**2+y**2);
       set xrange [\xmin:\xmax];
       set yrange [-\ymax:\ymax];
       set view 0,0;
       set isosample 100,100;
       set cont base;
       set cntrparam levels discrete \C;
       unset surface;
       splot f(x,y)
    };
  
    \message{Intersections...}
    \path[name intersections={of=Elines and xline,total=\t}]
      \pgfextra{\xdef\Nb{\t}};
    \message{ found \Nb ^^J}
    \foreach \i in {1,...,\Nb}{
      \message{  \i}
      \xy{(intersection-\i)}{\xpt}{\ypt}
      \EVector{\xpt}{\ypt}
      \message{ (\D,\x,\y)^^J}
    }
}

%flusso
\usepackage[outline]{contour} % glow around text
\usetikzlibrary{angles,quotes} % for pic (angle labels)
\usetikzlibrary{decorations.markings}
\usetikzlibrary{shapes} % for path name
\tikzset{>=latex} % for LaTeX arrow head
\contourlength{1.4pt}

\usepackage{xcolor}
\colorlet{Ecol}{orange!90!black}
\colorlet{EcolFL}{orange!90!black}
\colorlet{veccol}{green!45!black}
\colorlet{EFcol}{red!60!black}
\tikzstyle{charged}=[top color=blue!20,bottom color=blue!40,shading angle=10]
\tikzstyle{darkcharged}=[very thin,top color=blue!60,bottom color=blue!80,shading angle=10]
\tikzstyle{charge+}=[very thin,top color=red!80,bottom color=red!80!black,shading angle=-5]
\tikzstyle{charge-}=[very thin,top color=blue!50,bottom color=blue!70!white!90!black,shading angle=10]
\tikzstyle{darkcharged}=[very thin,top color=blue!60,bottom color=blue!80,shading angle=10]
\tikzstyle{gauss surf}=[green!70!black,top color=green!2,bottom color=green!80!black!70,shading angle=5,fill opacity=0.6]
\tikzstyle{gauss lid}=[gauss surf,middle color=green!80!black!20,shading angle=40,fill opacity=0.7]
\tikzstyle{gauss dark}=[green!60!black,fill=green!60!black!70,fill opacity=0.8]
\tikzstyle{gauss line}=[green!80!black]
\tikzstyle{gauss dashed line}=[green!60!black!80,dashed,line width=0.2]
\tikzstyle{vector}=[->,thick,veccol]
\tikzstyle{normalvec}=[->,thick,blue!80!black!80]
\tikzstyle{EField}=[->,thick,Ecol]
\tikzstyle{EField dashed}=[dashed,Ecol,line width=0.6]
\tikzset{
  EFieldLine/.style={thick,EcolFL,decoration={markings,
                     mark=at position #1 with {\arrow{latex}}},
                     postaction={decorate}},
  EFieldLine/.default=0.5}
\tikzstyle{measure}=[fill=white,midway,outer sep=2]

%legge di gauss
\usepackage[outline]{contour} % glow around text
\usetikzlibrary{angles,quotes} % for pic (angle labels)
\usetikzlibrary{decorations.markings}
\usetikzlibrary{shapes,intersections} % for path name
\tikzset{>=latex} % for LaTeX arrow head
\contourlength{1.8pt}

\usepackage{xcolor}
\colorlet{Ecol}{orange!90!black}
\colorlet{EcolFL}{orange!80!black}
\colorlet{veccol}{green!45!black}
\colorlet{EFcol}{red!60!black}
\colorlet{pluscol}{red!60!black}
\colorlet{minuscol}{blue!60!black}
\colorlet{gausscol}{green!50!black!80}
\tikzstyle{charged}=[top color=blue!20,bottom color=blue!40,shading angle=10]
\tikzstyle{charge+}=[very thin,draw=black,top color=red!80,bottom color=red!80!black,shading angle=-5]
\tikzstyle{charge-}=[very thin,draw=black,top color=blue!50,bottom color=blue!70!white!90!black,shading angle=10]
\tikzstyle{darkcharged}=[very thin,top color=blue!60,bottom color=blue!80,shading angle=10]
\tikzstyle{gauss surf}=[green!40!black,top color=green!2,bottom color=green!80!black!70,shading angle=5,fill opacity=0.6]
\tikzstyle{gauss dark}=[green!40!black,fill=green!40!black!70,fill opacity=0.8]
\tikzstyle{gauss line}=[green!40!black]
\tikzstyle{gauss dashed line}=[green!60!black!80,dashed,line width=0.2]
\tikzstyle{vector}=[->,thick,veccol]
\tikzstyle{normalvec}=[->,thick,blue!80!black!80]
\tikzstyle{EField}=[->,thick,Ecol]
\tikzstyle{EField dashed}=[dashed,Ecol,line width=0.6]
\tikzset{
  EFieldLine/.style={thick,EcolFL,decoration={markings,
                     mark=at position #1 with {\arrow{latex}}},
                     postaction={decorate}},
  EFieldLine/.default=0.5}
\tikzstyle{metal}=[top color=black!5,bottom color=black!15,shading angle=30]
\tikzstyle{measure}=[fill=green!70!black!8,midway,outer sep=0,inner sep=1]


%filo elettrico
\usepackage{xcolor}
\colorlet{Ecol}{orange!90!black}
\colorlet{EcolFL}{orange!80!black}
\colorlet{veccol}{green!45!black}
\colorlet{EFcol}{red!60!black}
\tikzstyle{charged}=[top color=blue!20,bottom color=blue!40,shading angle=10]
\tikzstyle{darkcharged}=[very thin,top color=blue!60,bottom color=blue!80,shading angle=10]
\tikzstyle{charge+}=[very thin,top color=red!80,bottom color=red!80!black,shading angle=-5]
\tikzstyle{charge-}=[very thin,top color=blue!50,bottom color=blue!70!white!90!black,shading angle=10]
\tikzstyle{gauss surf}=[green!40!black,top color=green!2,bottom color=green!80!black!70,shading angle=5,fill opacity=0.5]
\tikzstyle{gauss lid}=[gauss surf,middle color=green!80!black!20,shading angle=40,fill opacity=0.6]
\tikzstyle{gauss dark}=[green!50!black,fill=green!60!black!70,fill opacity=0.8]
\tikzstyle{gauss line}=[green!40!black]
\tikzstyle{gauss dashed line}=[green!60!black!80,dashed,line width=0.2]
\tikzstyle{EField}=[->,thick,Ecol]
\tikzstyle{vector}=[->,thick,veccol]
\tikzstyle{normalvec}=[->,thick,blue!80!black!80]
\tikzstyle{EFieldLine}=[thick,EcolFL,decoration={markings,
          mark=at position 0.5 with {\arrow{latex}}},
          postaction={decorate}]
\tikzstyle{measure}=[fill=white,midway,outer sep=2]


%per i grafici campo elettrico

\usepackage{amsmath} % for \dfrac
\usepackage{physics}
\usepackage{tikz,pgfplots}
\usetikzlibrary{angles,quotes} % for pic (angle labels)
\usetikzlibrary{decorations.markings}
\tikzset{>=latex} % for LaTeX arrow head
\usepackage{xcolor}
\colorlet{Ecol}{orange!90!black}
\colorlet{veccol}{green!45!black}
\tikzstyle{EFieldd}=[thick,Ecol]


%parallelepipedo cilindro infinitesimo
\usepackage{tikz-3dplot}

%Disegno condensatore in %\section{Il condensatore}
\usepackage{mathtools}
\usetikzlibrary{decorations.markings}

\colorlet{Ecolor}{orange!90!black}
\colorlet{pluscolor}{red!60!black}
\colorlet{minuscolor}{blue!60!black}
\tikzstyle{anode}=[top color=red!20, bottom color=red!50]
\tikzstyle{cathode}=[top color=blue!20, bottom color=blue!40]
\tikzstyle{charge+}=[very thin,top color=red!50, bottom color=red!80]
\tikzstyle{charge-}=[very thin,top color=blue!40, bottom color=blue!70]
\tikzset{EFieldLine/.style={
  Ecolor, decoration={markings, mark=at position #1 with {\arrow{stealth}}}, postaction={decorate}}
}

%condensatore
\usepackage{bm}
\usepackage{physics}
\usepackage{tikz,pgfplots}
\usetikzlibrary{angles,quotes} % for pic (angle labels)
\usetikzlibrary{decorations.markings}
\usetikzlibrary{calc}
\tikzset{>=latex} % for LaTeX arrow head

\usepackage{xcolor}
\colorlet{Ecol}{orange!90!black}
\colorlet{EcolFL}{orange!90!black}
\colorlet{veccol}{green!45!black}
\colorlet{EFcol}{red!60!black}
\colorlet{pluscol}{red!60!black}
\colorlet{minuscol}{blue!60!black}
\tikzstyle{anode}=[top color=red!20,bottom color=red!50,shading angle=20]
\tikzstyle{cathode}=[top color=blue!20,bottom color=blue!40,shading angle=20]
\tikzstyle{charge+}=[very thin,top color=red!80,bottom color=red!80!black,shading angle=-5]
\tikzstyle{charge-}=[very thin,top color=blue!50,bottom color=blue!70!white!90!black,shading angle=10]
\tikzstyle{vector}=[->,thick,veccol]
\tikzstyle{normalvec}=[->,thick,blue!80!black!80]
\tikzstyle{Cstyle}=[very thick,orange!90!black]
\tikzstyle{EField}=[->,thick,Ecol]
\tikzstyle{EField dashed}=[dashed,Ecol,line width=0.6]
\tikzset{
  EFieldLine/.style={thick,EcolFL,decoration={markings,
                     mark=at position #1 with {\arrow{latex}}},
                     postaction={decorate}},
  EFieldLine/.default=0.5}
\tikzstyle{measure}=[fill=white,midway,outer sep=2]




%Grafici condensaori e resistori
\usepackage{tikz,pgfplots}
\usepackage[siunitx]{circuitikz} %[symbols]
\usepackage{xcolor}
\tikzset{>=latex} % for LaTeX arrow head
\colorlet{Icol}{blue!50!black}
\colorlet{Ccol}{orange!90!black}
\colorlet{pluscol}{red!60!black}
\colorlet{minuscol}{blue!60!black}
%\tikzstyle{charged}=[top color=blue!20,bottom color=blue!40,shading angle=10]
\tikzstyle{thick C}=[C,thick,color=Ccol,Ccol,l=$C$]
\tikzstyle{mybattery}=[battery1,l=$\Delta V$,invert]

%forza elettrica vs forza gravitazionale

%\usepackage{bm} % \bm
%\usepackage{physics}
%\usepackage{tikz,pgfplots}
%\usepackage[outline]{contour} % glow around text
%\usetikzlibrary{angles,quotes} % for pic (angle labels)
%\usetikzlibrary{calc}
%\usetikzlibrary{decorations.markings}
%\tikzset{>=latex} % for LaTeX arrow head
%\contourlength{1.6pt}
%\usepackage{xcolor}
%\colorlet{Ecol}{orange!90!black}
%\colorlet{EcolFL}{orange!80!black}
%\colorlet{veccol}{green!45!black}
%\colorlet{EFcol}{red!60!black}
%\tikzstyle{EcolEP}=[blue!80!white]
%\tikzstyle{charged}=[top color=blue!30,bottom color=blue!50,shading angle=10]
%\tikzstyle{darkcharged}=[very thin,top color=blue!60,bottom color=blue!80,shading angle=10]
%\tikzstyle{charge+}=[very thin,top color=red!50,bottom color=red!90!black,shading angle=20]
%\tikzstyle{charge-}=[very thin,top color=blue!50,bottom color=blue!80,shading angle=20]
%\tikzstyle{gauss surf}=[blue!90!black,top color=blue!2,bottom %color=blue!80!black!70,shading angle=5,fill opacity=0.1]
%\tikzstyle{gauss line}=[blue!90!black]
%\tikzstyle{vector}=[->,very thick,veccol]
%\tikzset{EFieldLine/.style={thick,EcolFL,decoration={markings,mark=at position #1 with %{\arrow{latex}}},
%                                 postaction={decorate}},
%         EFieldLine/.default=0.5}
%\tikzstyle{measure}=[fill=white,midway,outer sep=2]



%Corrente
\usepackage{physics}
\usepackage{bm}
\usepackage{tikz}
\tikzset{>=latex} % for LaTeX arrow head
\usetikzlibrary{decorations.markings}
\usepackage{xcolor}
\colorlet{Ecol}{orange!90!black}
%\colorlet{charge+}{blue!80!white}
\tikzstyle{charge0}=[top color=green!80!black!50,bottom color=green!80!black,shading angle=20]
\tikzstyle{charge+}=[top color=red!50,bottom color=red!70!black,shading angle=20]
\tikzstyle{charge-}=[top color=blue!50,bottom color=blue!80,shading angle=20]
\tikzstyle{metal}=[top color=black!15,bottom color=black!25,middle color=black!5,shading angle=10]
\tikzset{
  EField/.style={thick,Ecol,decoration={markings,
                 mark=at position #1 with {\arrow{latex}}},
                 postaction={decorate}},
  EField/.default=0.5}


%$campo elettromotore$
\colorlet{Icol}{blue!50!black}
\colorlet{Ccol}{orange!90!black}
\colorlet{pluscol}{red!60!black}
\colorlet{minuscol}{blue!60!black}
%\tikzstyle{charged}=[top color=blue!20,bottom color=blue!40,shading angle=10]
\tikzstyle{thick C}=[C,thick,color=Ccol,Ccol,l=$C$]
\tikzstyle{mybattery}=[battery1,l=$\Delta V$,invert]


%circuiti elettrici
\usepackage[siunitx]{circuitikz} %[symbols]
\usepackage[outline]{contour} % glow around text
\usetikzlibrary{arrows,arrows.meta}
\usetikzlibrary{decorations.markings}
\tikzset{>=latex} % for LaTeX arrow head
\usepackage{xcolor}
\colorlet{Icol}{blue!50!black}
\colorlet{Ccol}{orange!90!black}
\colorlet{Rcol}{green!50!black}
\colorlet{loopcol}{red!90!black!25}
\colorlet{pluscol}{red!60!black}
\colorlet{minuscol}{blue!60!black}
\newcommand\EMF{\mathcal{E}} %\varepsilon}
\contourlength{1.5pt}
\tikzstyle{EMF}=[battery1,l=$\EMF$,invert]
\tikzstyle{internal R}=[R,color=Rcol,Rcol,l=$r$,/tikz/circuitikz/bipoles/length=30pt]
\tikzstyle{loop}=[->,red!90!black!25]
\tikzstyle{loop label}=[loopcol,fill=white,scale=0.8,inner sep=1]
\tikzstyle{thick R}=[R,color=Rcol,thick,Rcol,l=$R$]
\tikzstyle{thick R1}=[R,color=Rcol,thick,Rcol,l=$R1$]
\tikzstyle{thick R2}=[R,color=Rcol,thick,Rcol,l=$R2$]
\tikzstyle{thick R3}=[R,color=Rcol,thick,Rcol,l=$R3$]
\tikzstyle{thick C}=[C,thick,color=Ccol,Ccol,l=$C$]
\tikzstyle{thick C1}=[C,thick,color=Ccol,Ccol,l=$C1$]
\tikzstyle{myswitch}=[closing switch,line width=0.3] %-{Latex[length=3]},
\tikzstyle{myswitchi}=[opening switch,line width=0.3, invert] %-{Latex[length=3]},




%ancora circuiti ma con elelttromotore
\usepackage[siunitx]{circuitikz} %[symbols]
\usepackage[outline]{contour} % glow around text
\usetikzlibrary{arrows,arrows.meta}
\usetikzlibrary{decorations.markings}
\tikzset{>=latex} % for LaTeX arrow head
\usepackage{xcolor}
\colorlet{Icol}{blue!50!black}
\colorlet{Ccol}{orange!90!black}
\colorlet{Rcol}{green!50!black}
\colorlet{Lcol}{violet!90}
\colorlet{loopcol}{red!90!black!25}
\colorlet{pluscol}{red!60!black}
\colorlet{minuscol}{blue!60!black}
\contourlength{1.5pt}
%\tikzstyle{EMF}=[battery1,l=$\AC_0$,invert]
\tikzstyle{AC}=[sV,/tikz/circuitikz/bipoles/length=25pt,l=$\EMF(t)$]
\tikzstyle{internal R}=[R,color=Rcol,Rcol,l=$r$,/tikz/circuitikz/bipoles/length=30pt]
\tikzstyle{loop}=[->,red!90!black!25]
\tikzstyle{loop label}=[loopcol,fill=white,scale=0.8,inner sep=1]
\tikzstyle{thick R}=[R,color=Rcol,thick,Rcol,l=$R$]
\tikzstyle{thick C}=[C,thick,color=Ccol,Ccol,l=$C$]
\tikzstyle{thick L}=[L,thick,color=Lcol,Lcol,l=$L$,/tikz/circuitikz/bipoles/length=56pt] %inductor
\tikzstyle{thick Z}=[generic,color=Icol,thick,Icol,l=$Z$,fill=Icol!6]


%legge di Kirchhoff
\usepackage[siunitx]{circuitikz} %[symbols]
\usepackage[outline]{contour} % glow around text
\usetikzlibrary{arrows}
\usetikzlibrary{decorations.markings}
\tikzset{>=latex} % for LaTeX arrow head
\usepackage{xcolor}
\colorlet{Icol}{blue!50!black}
\colorlet{Ccol}{orange!90!black}
\colorlet{Rcol}{green!50!black}
\colorlet{loopcol}{red!90!black!25}
\colorlet{pluscol}{red!60!black}
\colorlet{minuscol}{blue!60!black}
%\tikzstyle{charged}=[top color=blue!20,bottom color=blue!40,shading angle=10]
\contourlength{1.5pt}
\tikzstyle{EMF}=[battery1,l=$\EMF$,invert]
\tikzstyle{internal R}=[R,color=Rcol,Rcol,l=$r$,/tikz/circuitikz/bipoles/length=30pt]
\tikzstyle{loop}=[->,red!90!black!25]
\tikzstyle{loop label}=[loopcol,fill=white,scale=0.8,inner sep=1]
\tikzstyle{thick R}=[R,color=Rcol,thick,Rcol,l=$R$]
%\tikzset{
%  loop/.style={thick,red!80!black!30,decoration={markings,
%               mark=at position #1 with {\arrow{latex}}},
%               postaction={decorate}},
%  loop/.default=0.6}


%Carica di un consensatore circuito RC

\colorlet{Rcol}{green!60!black}
\colorlet{myblue}{blue!70!black}
\colorlet{myred}{red!70!black}
\colorlet{Ecol}{orange!90!black}
\tikzstyle{Rline}=[Rcol,thick]
\tikzstyle{gline}=[Rcol,thick]
\tikzstyle{bline}=[myblue,thick]
\tikzstyle{rline}=[myred,thick]




%campo elettromagentico

\usepackage{tikz,pgfplots}
\usepackage{tikz-3dplot}
\usepackage[outline]{contour} % glow around text
\usetikzlibrary{angles,quotes} % for pic (angle labels)
\usetikzlibrary{arrows,arrows.meta}
\usetikzlibrary{calc}
\usetikzlibrary{decorations.markings}
\tikzset{>=latex} % for LaTeX arrow head
\usepackage{xcolor}
\colorlet{veccol}{green!45!black}
\colorlet{Bcol}{violet!90}
\colorlet{BFcol}{red!70!black}
\colorlet{veccol}{green!45!black}
\colorlet{Icol}{blue!70!black}
\colorlet{Ampcol}{green!60!black!70}
\tikzstyle{BField}=[->,thick,Bcol]
\tikzstyle{current}=[->,Icol,thick]
\tikzstyle{force}=[->,thick,BFcol]
\tikzstyle{vector}=[->,thick,veccol]
\tikzstyle{velocity}=[->,very thick,vcol]
\tikzstyle{charge+}=[very thin,draw=black,top color=red!50,bottom color=red!90!black,shading angle=20,circle,inner sep=0.5]
\tikzstyle{charge-}=[very thin,draw=black,top color=blue!50,bottom color=blue!80,shading angle=20,circle,inner sep=0.5]
\tikzstyle{metal}=[line width=0.3,top color=black!15,bottom color=black!25,middle color=black!20,shading angle=10]
\tikzstyle{darkmetal}=[line width=0.4,top color=red!20!black!40,bottom color=red!20!black!70,middle color=red!20!black!30,shading angle=10]
\tikzstyle{measline}=[{Latex[length=3]}-{Latex[length=3]}]
\tikzset{
  BFieldLine/.style={thick,Bcol,decoration={markings,mark=at position #1 with {\arrow{latex}}},
                                 postaction={decorate}},
  BFieldLine/.default=0.5,
  Ampcurve/.style={thick,Ampcol,decoration={markings,mark=at position #1 with {\arrow{latex}}},
                                postaction={decorate}},
  Ampcurve/.default=0.55,
  pics/Bin/.style={
    code={
      \def\R{0.12}
      \draw[pic actions,line width=0.6,#1,fill=white] % ,thick
        (0,0) circle (\R) (-135:.75*\R) -- (45:.75*\R) (-45:.75*\R) -- (135:.75*\R);
  }},
  pics/Bout/.style={
    code={
      \def\R{0.12}
      \draw[pic actions,line width=0.6,#1,fill=white] (0,0) circle (\R);
      \fill[pic actions,#1] (0,0) circle (0.3*\R);
  }},
  pics/Bin/.default=Bcol,
  pics/Bout/.default=Bcol,
}
\tikzstyle{measure}=[fill=white,midway,outer sep=2]
\contourlength{1.4pt}

% RING SHADING
\makeatletter
\pgfdeclareradialshading[tikz@ball]{ring}{\pgfpoint{0cm}{0cm}}%
{rgb(0cm)=(1,1,1);
rgb(0.719cm)=(1,1,1);
color(0.72cm)=(tikz@ball);
rgb(0.9cm)=(1,1,1)}
\tikzoption{ring color}{\pgfutil@colorlet{tikz@ball}{#1}\def\tikz@shading{ring}\tikz@addmode{\tikz@mode@shadetrue}}
\makeatother

%solenoide

\usepackage{ifthen}
\usepackage{tikz,pgfplots}
\usepackage{tikz-3dplot}
\usepackage{auto-pst-pdf}
\usepackage{pst-magneticfield}
\usepackage[outline]{contour} % glow around text
\usetikzlibrary{angles,quotes} % for pic (angle labels)
\usetikzlibrary{arrows,arrows.meta}
\usetikzlibrary{calc}
\usetikzlibrary{decorations.markings}
\tikzset{>=latex} % for LaTeX arrow head
\usepackage{xcolor}
\colorlet{Ecol}{orange!90!black}
\colorlet{EcolFL}{orange!80!black}
\colorlet{veccol}{green!45!black}
\colorlet{projcol}{blue!70!black}
\colorlet{EFcol}{red!60!black}
\colorlet{Bcol}{violet!90}
\colorlet{Bcol1}{violet!80!blue!90}
\colorlet{Bcol2}{violet!80!red!90}
\colorlet{BFcol}{red!70!black}
\colorlet{veccol}{green!45!black}
\colorlet{Icol}{blue!70!black}
\colorlet{Ampcol}{green!60!black!70}
\tikzstyle{BField}=[->,thick,Bcol]
\tikzstyle{current}=[->,Icol,thick]
\tikzstyle{force}=[->,thick,BFcol]
\tikzstyle{vector}=[->,thick,veccol]
\tikzstyle{velocity}=[->,very thick,vcol]
\tikzstyle{metal}=[top color=black!15,bottom color=black!25,middle color=black!20,shading angle=10]
\tikzstyle{darkmetal}=[top color=black!40,bottom color=black!70,middle color=black!30,shading angle=10]
\tikzstyle{lightmetal}=[thin,black!20,top color=black!3,bottom color=black!6,middle color=black!1,shading angle=10]
\tikzstyle{proj}=[projcol!80,line width=0.08] %very thin
\tikzstyle{area}=[draw=veccol,fill=veccol!80,fill opacity=0.6]
\tikzstyle{measline}=[{Latex[length=3]}-{Latex[length=3]}]
\tikzset{
  BFieldLine/.style={thick,Bcol,decoration={markings,mark=at position #1 with {\arrow{latex}}},
                                postaction={decorate}},
  BFieldLine/.default=0.5,
  Ampcurve/.style={Ampcol,decoration={markings,mark=at position #1 with {\arrow{latex}}},
                          postaction={decorate}},
  Ampcurve/.default=0.55,
  pics/Bin/.style={
    code={
      \def\R{0.12}
      \draw[pic actions,line width=0.6,#1,fill=white] % ,thick
        (0,0) circle (\R) (-135:.75*\R) -- (45:.75*\R) (-45:.75*\R) -- (135:.75*\R);
  }},
  pics/Bout/.style={
    code={
      \def\R{0.12}
      \draw[pic actions,line width=0.6,#1,fill=white] (0,0) circle (\R);
      \fill[pic actions,#1] (0,0) circle (0.3*\R);
  }},
  pics/Bin/.default=Bcol,
  pics/Bout/.default=Bcol,
}
\tikzstyle{measure}=[fill=white,midway,outer sep=2]
%\newcommand\arrmark[1]{mark=at position #1 with {\arrow{latex}}}
%\def\myarrmark#1{mark=at position #1 with {\arrow{latex}}}
\contourlength{1.4pt}

% RING SHADING
\makeatletter
\pgfdeclareradialshading[tikz@ball]{ring}{\pgfpoint{0cm}{0cm}}%
{rgb(0cm)=(1,1,1);
rgb(0.719cm)=(1,1,1);
color(0.72cm)=(tikz@ball);
rgb(0.9cm)=(1,1,1)}
\tikzoption{ring color}{\pgfutil@colorlet{tikz@ball}{#1}\def\tikz@shading{ring}\tikz@addmode{\tikz@mode@shadetrue}}
\makeatother






\newcommand{\myvoltmeter}[2] 
{  % #1 = name , #2 = rotation angle
  \begin{scope}[transform shape,rotate=#2]
  \draw[thick] (#1)node(){$\mathbf V$} circle (11pt);
  \draw[rotate=45,-latex] (#1)  +(-17pt,0) --+(17pt,0);
  \end{scope}
}